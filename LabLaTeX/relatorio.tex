\documentclass[a4paper, 11pt]{article}
\usepackage{geometry}
\usepackage{indentfirst}
\usepackage{setspace}
\usepackage{amsmath}
\usepackage{amssymb}
\usepackage{graphicx}
\usepackage{wrapfig}
\usepackage{caption}
\usepackage{indentfirst}
\setlength{\parindent}{20pt}
\usepackage{amssymb}
\usepackage{float}
\usepackage{subcaption}

\graphicspath{ {./images/} }
\geometry{left=2.5cm, right=2.5cm, top=2.5cm, bottom=2.5cm}

\begin{document}	
	\title{Lab Exercise Part 1 e 2?? }
	\author{{\small Alexandre Rodrigues (2039952)}}
	\date{\today}
	\maketitle
	
	\section{Introduction}
		$\ldots$
		
	\section{Technical Approach}
		
		
		\subsection{Instancing}
			$\ldots$
			Class 1 - Random
			
			Class 2 - Domain specific random
				How...
				
		\subsection{part2}
			Used tabu search or genetic or various....
			Used from moodle?? Added: clever initial solution, intesification, diversification, 3 opt instead of 2 opt, alterating $\ldots$	
			
	
	\section{Results}
		I tested class 3 only for n=10 with 6 instances, 5 runs each.
		The class 2 intances are random but slightly domain specific.
		In this case I simply removed the possiblity to have holes in the borders of the board.
		
		Class 1 is fully random:
		board as size = N
		so $ x \in {0,N-1}$
		hole of size 1
		
		Both class 1 and 2 were tested using 30 random instances for $ 10 \le n \le 70 $.
		Due to the considerable time I reduced it to 10 instances for $ 80 \le n \le 100 $.
		
		
		The time to drill a hole is constant so we can diregard it.
		The total cost would be $ cost_{real}  = cost_{exp} + cN, c\in \mathbb{R}$.
		
		The cost matrix was computed from the hole positions (random or not) using Manhattan distance
	
		
		\begin{table}[H]
			\centering
			\begin{tabular}{c|c|c}
				\textbf{$ n $} 	& \textbf{class1} 	& \textbf{class2}  \\ \hline
				$ 10  $			& $ 0.125 s $ 		& $ 0.131 s $ \\ \hline
				$ 20  $			& $ 0.537 s $ 		& $ 0.449 s $ \\ \hline
				$ 30  $			& $ 1.561 s $	 	& $ 1.580 s $ \\ \hline
				$ 40  $			& $ 5.549 s $ 		& $ 6.303 s $ \\ \hline
				$ 50  $			& $ 14.534 s $ 		& $ 13.757 s $ \\ \hline
				$ 60 $			& $ 29.168 s $ 		& $ 26.621 s $ \\ \hline
				$ 70 $			& $ 47.808 s $	 	& $ 48.775 s $ \\ \hline
				$ 80 $			& $ 91.089 s $ 		& $ 105.980 s $ \\ \hline
				$ 90 $			& $ 142.739 s $	 	& $ 199.926 s $ \\ \hline
				$ 100 $			& $ 257.982 s $ 		& $ 292.470 s $ \\ 
			\end{tabular}
			\caption{Average Time}
			\label{table:times}
		\end{table}
		
		Assuming a max time as 20 seconds.... we can solve for up to ?? nodes.
		
		Class 1 vs Class 2 $\ldots$
		
		Tested 3,4,5 instance for each n and $\ldots$
	\begin{table}[H]
		\centering
		\begin{tabular}{c|c|c}
			\textbf{$ n $} 	& \textbf{class1} & \textbf{class2}  \\ \hline
			$ 5  $			& $ 0.15 $ 			& $  $ \\ \hline
			$ 10  $			& $ 0.25 $ 			& $ $ \\ \hline
			$ 20  $			& $  $ 			& $  $ \\ \hline
			$ 30  $			& $  $	 		& $  $ \\ \hline
			$ 50  $			& $  $	 		& $  $ \\ \hline
			$ 70  $			& $  $	 		& $  $ \\ \hline
			$ 100 $			& $  $	 		& $  $ \\ \hline
			$ 150 $			& $  $	 		& $  $ \\ \hline
			$ 200 $			& $  $ 			& $  $ \\ 
		\end{tabular}
		\caption{Average Time}
		\label{table:times2}
	\end{table}

	\begin{table}[H]
		\centering
		\begin{tabular}{c|c|c|c}
			\textbf{$ n $} 	& \textbf{sol} & \textbf{optimal}   & \textbf{cplex heuristic}  \\ \hline
			$ 5  $			& $ 40 $ 		& $ 30 $ 			& N/A \\ \hline
			$ 10  $			& $ 100 $ 		& $ 90 $ 			& N/A \\ \hline
			$ 20  $			& $  $ 			& $  $ 				& N/A \\ \hline
			$ 30  $			& $  $	 		& $  $ 				& N/A \\ \hline
			$ 50  $			& $  $	 		& $  $ 				& $  $ \\ \hline
			$ 70  $			& $  $	 		& $  $ 				& $  $ \\ \hline
			$ 100 $			& $  $	 		& $  $ 				& $  $ \\ \hline
			$ 150 $			& $  $	 		& $  $ 				& $  $ \\ \hline
			$ 200 $			& $  $ 			& $  $ 				& $  $ \\
		\end{tabular}
		\caption{Solution}
		\label{table:sols}
	\end{table}
	
	Assuming a max time as 20 seconds.... we can solve for up to ?? nodes.
	
	Class 1 vs Class 2 $\ldots$
	
	std. deviation of $\ldots$
	
	
	\section{Conclusions}
	
	
\end{document}



